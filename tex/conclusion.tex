\section{Conclusion}

In this work, we showed that high mutational load derails the evolution of cooperative, multicellular phenotypes.
Specifically, we were able to observe,
\begin{itemize}
\item cooperative reproductive behavior observed under low mutational load is replaced by antagonistic reproductive behavior under high mutational load,
\item cooperative resource sharing behavior observed under low mutational load disappears under higher mutational loads,
\item the evolution of cooperative reproductive behavior in numerically large cell groups (e.g., large resource wave treatments) is more sensitive to mutational load than in numerically small cell groups (e.g., small resource wave treatments), and
\item the evolution of resource sharing in small cell groups is more sensitive to mutational load than in large cell groups.
\end{itemize}

Questions remain, especially due to small number of cellular generations elapsed in some treatments.
Performing longer evolutionary runs, at least until descendants from a single seeded genome sweep the population, would provide certainty that our observations --- especially with respect large resource wave treatments --- are not artifacts of small generation counts.
Performing much longer evolutionary runs would provide another opportunity: to look for the evolution of behaviors that counteract mutational load so that cooperative strategies can succeed despite mutation load.
Biological multicellular organisms have evolved mechanisms to manage somatic mutation and, especially, reduce the incidence of malignant mutants (e.g., cancer).
Strategies include genetic pathways that induce a cell to destroy itself in response to genetic damage \citep{lee1995apoptosis} and immune responses where other cells recognize and destroy malignant cells \citep{Martinez-Lostao5047}.
To attempt to evolve strategies to counteract mutational load we might begin with a population under low mutational load where cooperative behavior evolves and then gradually increase mutational load.
If cooperative behavior remains at mutational loads where it does not spontaneously evolve, this might suggest the presence of mutation-fighting adaptation.
The mechanism of such an adaptation might be of scientific or practical interest.

DISHTINY is designed with in a distributed framework in which individual digital cells operate independently and periodically communicate with neighbors via messages.
Hence, it can scale as additional computing resources are provided.
Parallel computing is widely exploited in evolutionary computing, where subpopulations are farmed out for periods of isolated evolution or single genotypes are farmed out for fitness evaluation
\citep{lin1994coarse, real17a}.
DISHTINY presents a more fundamental parallelization potential: principled parallelization of the evolving individual phenotype at arbitrary scale (i.e., a high-level individual as a large collection of individual cells distributed across processing elements).
Such parallelization will be key to realizing evolving computational systems with scale --- and, perhaps, complexity --- approaching those of biological systems.
From an applied perspective, such distributed systems might enable the evolution of more complex digital organisms that exhibit more sophisticated, and potentially useful, capabilities (for example, as reinforcement-learning agents).
From a scientific perspective, we believe that as an experimental platform large-scale digital multicellularity will yield unique insight into questions raised by theoretical evolutionary biology with respect to evolutionary transitions in individuality.
