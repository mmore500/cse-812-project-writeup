\section{Introduction}

In a fraternal evolutionary transition of individuality, a new, more complex replicating entity is derived from the combination of cooperating kin which have entwined their long-term fates \citep{west2015major}.
Eusocial insect colonies and multicellular organisms exemplify this phenomenon  \citep{smith1997major}.
In the first case, individual insects (e.g., ants) act in concert towards the subsistence and propagation of the collective (e.g., an ant colony).
In the second, individual cells join together into a highly structured collective in which only a fraction participate in the germ line.

A meta-organism composed of fraternal kin (e.g., a multicellular organism composed of individual cells) in some ways resembles a distributed system.
In order for a meta-organism to succeed, its constituent parts --- which interact asynchronously --- must cooperate to perform tasks and make decisions at the collective level.
For example, ants use pheromones to exchange information about the location and quality of food sources in order to collectively explore their environment and forage resources for the colony.

Distributed systems must contend with failures and, ideally, nevertheless continue to function effectively.
The most commonly considered failure mode is crash failures, where an element of the distributed system loses some functionality or becomes unresponsive.
However, failure modes where a node begins to exhibit arbitrary, or even malicious, behavior --- referred to as byzantine failures --- are also possible.
Like distributed systems, fraternal groups contend with failures of their constituent members.
Over time, an individual member accumulates mutations which, eventually, may cause that member to cease to function properly (i.e., a crash failure) or even act maliciously (i.e., a byzantine failure).
In particular, multicellular organisms often contend with cancer, a byzantine failure where tumor cells engage in unchecked reproduction.
Studying the relationship between mutation and fraternal transitions of individuality is of both scientific and practical interest.
From a scientific perspective, we might gain insight into the mutation-management challenges that fraternal groups face and the evolutionary trajectories that clear those hurdles.
From a practical perspective, better understanding evolved mutation-management solutions in fraternal groups may provide inspiration for new methods to secure distributed systems against failures.

In previous work, we have developed a digital framework to study fraternal transitions of individuality \cite{moreno2018toward}.
This framework, DISHTINY, is designed as distributed system: individual digital cells operate independently and periodically communicate with neighbors via messages.
This project uses DISHTINY to investigate the relationship between mutation intensity and the evolution of cooperation.
Specifically, we ask
\begin{itemize}
\item Do cooperative strategies become disadvantageous under higher mutation intensity?
\item What is the relationship between group size and the effect of mutation intensity on the viability of cooperative strategies?
\end{itemize}
