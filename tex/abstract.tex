\begin{abstract}

Natural evolutionary history is profoundly shaped by evolutionary transitions in individuality, episodes where independent replicating entities united to form more complex replicating entities.
Fraternal transitions in individuality occur specifically when the derived replicating entity is composed of kin groups of a lower-level replicator;
examples include the evolution of multicellularity and the evolution of eusocial insect colonies.
The conditions necessary for fraternal transitions to occur and the mechanisms by which such transitions occur have been a fruitful target of scientific interest and open questions that warrant continued investigation remain.

In previous work, we developed a digital framework to study fraternal transitions of individuality.
This framework, DISHTINY, is designed as distributed system: individual digital cells operate independently and periodically communicate with neighbors via messages.
This project uses DISHTINY to investigate the relationship between mutation intensity and the evolution of cooperation.
Specifically, we ask
\begin{itemize}
\item Do cooperative strategies become disadvantageous under higher mutation intensity?
\item What is the relationship between group size and the effect of mutation intensity on the viability of cooperative strategies?
\end{itemize}

We find that high mutational load derails the evolution of cooperative, multicellular phenotypes.
Specifically, we observed,
\begin{itemize}
\item cooperative reproductive behavior observed under low mutational load is replaced by antagonistic reproductive behavior under high mutational load,
\item cooperative resource sharing behavior observed under low mutational load disappears under higher mutational loads,
\item cooperative reproductive behavior and cooperative resource sharing exhibit opposite trends in sensitivity mutational intensity with respect to group size, the first being more sensitive in large groups and the latter being more sensitive in small groups.
\end{itemize}
\end{abstract}
